\documentclass[a4paper,12pt,reqno]{report}

% --------------------------------------------------------
% Packages
% --------------------------------------------------------
\usepackage{mathtools,graphicx}
\usepackage[colorlinks=true, allcolors=blue]{hyperref}
\usepackage[utf8]{inputenc}
\usepackage{amsmath,amsfonts,amssymb,amsthm,mathrsfs,bm}
\usepackage[margin=20mm]{geometry}
\usepackage{setspace}
\usepackage[table,xcdraw]{xcolor}
\usepackage{float} % 支持图片浮点嵌入 e.g.,\begin{figure}[H]
\usepackage{subfigure} % 支持多图
\usepackage{cite}
\usepackage{siunitx} % 支持各类正体单位 e.g.,\unit{\kg}(建议在数字和单位间插入\,以留出适当间隙)
\usepackage{txfonts} % 支持正体希腊字母 e.g.,$\muup$

% --------------------------------------------------------
% Custom Colours
% --------------------------------------------------------
\renewcommand{\baselinestretch}{1.5}    % 1.5-line spacing
\renewcommand{\rmdefault}{phv} % Arial
\renewcommand{\sfdefault}{phv} % Arial
\renewcommand\thesection{\arabic {section}} % Section numbers start from 1
\renewcommand{\bibname}{References}
\newcommand{\HRule}{\rule{\linewidth}{0.5mm}}
\DeclareUnicodeCharacter{2212}{-}
\renewcommand{\thefootnote}{**} 

% --------------------------------------------------------
% Open Page
% --------------------------------------------------------
\begin{document}
\input{./title.tex}

% --------------------------------------------------------
% Abstract
% --------------------------------------------------------
\vspace*{\fill}
\begin{flushleft}
    \emph{
        By submitting the work, I declare that:\\
        1. I have read and understood the University regulations relating to academic offences, including
        collusion and plagiarism
        \footnote{\href{http://www.qub.ac.uk/directorates/AcademicStudentAffairs/AcademicAffairs/GeneralRegulations/Procedures/ProceduresforDealingwithAcademicOffences/}{QUB Procedures For Dealing With Academic Offences}}
        \\
        2. The submission is my own original work and no part of it has been submitted for any other
        assignments, except as otherwise permitted;\\
        3. All sources used, published or unpublished, have been acknowledged;\\
        4. I give my consent for the work to be scanned using a plagiarism detection software.\\
    }
\end{flushleft}
\vspace*{\fill}
\thispagestyle{empty}

% --------------------------------------------------------
% Abstract
% --------------------------------------------------------
\begin{abstract}
    This paper designs a microcantilever made of silicon by calculation 
    and simulation. The microcantilever can measure the added mass on the microcantilever 
    by detecting the vibration frequency. And this paper simulates the change 
    of frequency by adding different masses and compares with the theoretical 
    calculation results to verify the reliability of the microcantilever measurement.
    At the end, this paper analyzes the relationship between the frequency of the 
    microcantilever and increasing or decreasing added mass of the microcantilever under 
    different added masses. The calculation and simulation results show that the resonant frequency 
    tends to decrease when the added mass increases, which is also consistent with the calculated formula of the resonant frequency.
    The microcantilever designed in this paper has the characteristics of micro size and weight
    and high sensitivity, which is suitable for various fields requiring MEMS sensor devices.
\end{abstract}

% --------------------------------------------------------
% Section 1: Introduction
% --------------------------------------------------------
\section{Introduction}
\label{sec:Introduction}
    \paragraph{Introduction of MEMS}

\newpage

% --------------------------------------------------------
% Section 2: Detailed description of the mass sensor design and analysis
% --------------------------------------------------------
\section{Detailed description of the mass sensor design and analysis}
\label{sec:Detailed description of the mass sensor design and analysis}
    % Section 2.1: Detailed description of the mass sensor design
    \subsection{Detailed description of the mass sensor design}
    \label{sec:Detailed description of the mass sensor design}
        Considering the length, width and thickness of the cantilever, let the thickness 
        $t=0.8\,\muup\unit{\m}$ and the resonant frequency $f=\qty{2.5}{\MHz}$.
        The resonant frequency in sensing mechanism is $f=\frac{1}{2 \pi}\sqrt{\frac{k_s}{m^*}}$, label to $f_1$, thus,
        $$f_1=\qty{2.5e6}{\hertz}=\qty{2500}{\kilo\hertz}$$
        \begin{equation}
            \frac{k_s}{m^*}=(5\pi\times10^6)^2 \tag{1.1}
        \end{equation}
        The resonant frequency when cantilever with bonded molecules is $f=\frac{1}{2\pi}\sqrt{\frac{k_s}{m^*+m_b^*}}$,  label to $f_2$.
        The design requirement is that the quality of $25\,\unit{\pg}$ can be detected at a change in resonant frequency of $5\%$, so we can calculate the resonant frequency when added mass:
        $$f_2=(1-5\%)\times\qty{2.5e6}=0.95\times\qty{2.5e6}{\hertz}=\qty{2375}{\kilo\hertz}$$
        Thus,
        \begin{equation}
            \frac{k_s}{m^*+m_b^*}=(0.95\times5\pi\times10^6)^2 \tag{1.2}
        \end{equation}
        Let (1.1)/(1.2),
        $$\frac{m^*}{m^*+m_b^*}=(0.95)^2$$
        According to design parameters, the added mass is $m_b^*=25\times10^{-15}$, so we can calculate the microcantilever effective mass:
        $$m^*=\frac{(0.95)^2\times m_b^*}{1-(0.95)^2}=\frac{(0.95)^2\times25\times10^{-15}}{1-(0.95)^2}=\qty{231.41e-15}{\kilo\gram}$$
        Using (1.1) to calculate the value of $k_s$:
        $$k_s=(5\pi\times10^6)^2\times m^*=(5\pi\times10^6)^2\times231.41\times10^{-15}=\qty{57.098}{\N/\m}$$
        We need to consider that the effective mass should be $\frac{2}{3}$ of the real mass, so we get $m^*=\frac{2}{3} \rho Lwt$. 
        And in order to simplify the calculation, assume that the density of silicon is $\qty{2300}{\kg\per\m^3}$. Thus,
        $$wL=\frac{3m^*}{2 \rho t}=\frac{3\times231.41\times10^{-15}}{2\times2300\times0.8\times10^{-6}}=0.18864\times10^{-9}$$
        The spring constant $k_s=\frac{2Ewt^3}{3L^3}$, in order to simplify the calculation, assume that Young's modulus $E=\qty{160}{\GPa}$. Thus,
        \begin{equation}
            \frac{w}{L^3}=\frac{3k_s}{2Et^3}=\frac{3\times57.098}{2\times160\times10^9\times(0.8\times10^{-6})^3}=1.0455\times10^9 \tag{1.3} \label{fomula.1.3}
        \end{equation}
        \begin{equation}
            L^4=\frac{0.18864\times10^{-9}}{1.0455\times10^{9}}=18.043\times10^{-20} \tag{1.4}
        \end{equation}
        Combine these two equations (1.3) and (1.4), we can calculate the value of $L$ and $w$:
        \begin{equation}
            \begin{cases}
                L=\qty{2.06e-5}=\qty{20.6e-6}{\m}=20.6\,\muup\unit{\m}\\
                w=\frac{0.18864\times10^{-9}}{2.06\times10^{-5}}=\qty{9.16e-6}{\m}=0.397\,\muup\unit{\m}
            \end{cases}
            \notag
        \end{equation}
        The effective mass of added mass should also be $\frac{2}{3}$ of the real mass, so we get $m_b^*=\frac{2}{3} \rho Lwt_m$. Then, we have known that the added mass $m_b^*$, so we can calculate the thickness of the added mass:
        $$t_m=\frac{3m_b^*}{2 \rho Lw}=\frac{3\times25\times10^{-15}}{2\times500\times20.6\times10^{-6}\times9.16\times10^{-6}}=\qty{0.397e-6}{\m}=0.397\,\muup\unit{\m}$$
        Finally, the parameter of canliever is shown below:
        \begin{table}[H]
            \centering
            \begin{tabular}{ccccccccccllll}
                \hline
                \textbf{Name}           &&&& \textbf{Expression}             &&&& \textbf{Value}          &&&& \textbf{Description}            \\ \hline
                $L$                     &&&& $20.6\,\muup\unit{\m}$            &&&& \qty{20.6e-6}{\m}       &&&& Cantilever Length               \\
                $w$                     &&&& $9.16\,\muup\unit{\m}$            &&&& \qty{9.16e-6}{\m}       &&&& Cantilever Width                \\
                $t_s$                   &&&& $0.8\,\muup\unit{\m}$             &&&& \qty{0.8e-6}{\m}        &&&& Silicon cantilever thickness    \\
                $t_m$                   &&&& $0.397\,\muup\unit{\m}$           &&&& \qty{0.397e-6}{\m}      &&&& Thickness of added mass         \\ \hline
            \end{tabular}
            \caption{Parameter}
            \label{Table.Parameter}
        \end{table}

    \newpage

    % Section 2.2: analysis
    \subsection{Analysis}
    \label{sec:Analysis}
        Using COMSOL, the eigenfrequency of simulation is: Eigenfrequency1 $=$ \qty{2385.5}{\kHz},\\
        Eigenfrequency2  $=$ \qty{2525.6}{\kHz}, Eigenfrequency3 $=11443\,\unit{\kHz}$ , Eigenfrequency4 $=11590\,\unit{\kHz}$.\\
        Thus, the resonant frequency of Microcantilever without added mass: $f_1=2525.6\,\unit{\kHz}$, 
        the resonant frequency of Microcantilever when the added mass ($25\,\unit{\pg}$): $f_2=2385.5\,\unit{\kHz}$.\\
        The figures of simulation are shown below:
        \begin{figure}[H]
            \centering
            \vspace{-0.35cm}
            \subfigtopskip=2pt
            \subfigbottomskip=2pt
            \subfigcapskip=-5pt
            \subfigure[Eigenfrequency 1]{
                \label{Eigenfrequency.sub.1}
                \includegraphics[width=0.32\linewidth]{figures/f2_normal.png}}
            \quad
            \subfigure[Eigenfrequency 2]{
                \label{Eigenfrequency.sub.2}
                \includegraphics[width=0.32\linewidth]{figures/f1_normal.png}}
            \\
            \subfigure[Eigenfrequency 3]{
                \label{Eigenfrequency.sub.3}
                \includegraphics[width=0.32\linewidth]{figures/f3_normal.png}}
            \quad
            \subfigure[Eigenfrequency 4]{
                \label{Eigenfrequency.sub.4}
                \includegraphics[width=0.32\linewidth]{figures/f4_normal.png}}
            \caption{Eigenfrequency of simulation}
            \label{Eigenfrequency}
        \end{figure}

        Compared to the values of calculation, it is obviously that they are different.
        In calculation, in order to simplify the calculation, Young's modulus is supposed 
        to be $160\,\unit{\GPa}$, and the density of silicon is supposed to be $2300\,\unit{\kg/\m^3}$. 
        On the contrary, these are the default allocation in COMSOL. So, it is normal 
        that the simulated values are inconsistent with the calculated results.
        The values of eigenfrequency in calculation and simulation is shown below:
        \begin{table}[H]
            \centering
            \begin{tabular}{ccccccccccc}
                \hline
                      &&&& \textbf{Calculation}   &&&& \textbf{Simulation}     &  \\ \hline
                $f_1$ &&&& $2500\,\unit{\kHz}$              &&&& $2525.6\,\unit{\kHz}$             &  \\
                $f_2$ &&&& $2375\,\unit{\kHz}$              &&&& $2385.5\,\unit{\kHz}$             &  \\ \hline
            \end{tabular}
            \caption{Comparing eigenfrequency}
            \label{Table.Comparing_f_between_c_s}
        \end{table}
        With calculation, the change in resonant frequency of simulation is:
        $$\frac{f_{1\_\mathrm{Sim}}-f_{2\_\mathrm{\mathrm{Sim}}}}{f_{1\_\mathrm{Sim}}}=\frac{2525.6-2385.5}{2525.6}=0.0547=5.47\% \textgreater 5\%$$
        It means that a $5.47\%$ change in resonant frequency is sufficient to enable detection of the mass ($25\,\unit{\pg}$), which is very close to the aim of design.\\
        Next, it is the analysis of the impact of increasing or decreasing added mass on frequency.
        \subsubsection{Decreasing the added mass}
        \paragraph{}
        If decreasing the added mass to $12.5\,\unit{\pg}$, due to $f_2=\frac{1}{2\pi}\sqrt{\frac{k_s}{m^*+m_b^*}}$, thus
        $$f_{2\_\mathrm{Dec}}=\frac{1}{2\pi}\sqrt{\frac{57.098}{231.41\times10^{-15}+12.5\times10^{-15}}}=0.024351\times10^8\,\unit{\Hz}=2435.1\,\unit{\kHz}$$
        \paragraph{}
        And with the simulation in COMSOL, the eigenfrequency($f_2$) changes to $2433.7\,\unit{\kHz}$. The simulation figure is shown below as Figure \ref{Fig.f2_decrease_add_mass}.
        \begin{figure}[H]
            \centering
            \includegraphics[width=0.7\textwidth]{figures/f2_decrease_add_mass}
            \caption{$f_2$ when decreasing the added mass}
            \label{Fig.f2_decrease_add_mass}
        \end{figure}

        \subsubsection{Increasing the added mass}
        \paragraph{}
        If increasing the added mass to $50\,\unit{\pg}$, due to $f_2=\frac{1}{2\pi}\sqrt{\frac{k_s}{m^*+m_b^*}}$, thus
        $$f_{2\_\mathrm{Inc}}=\frac{1}{2\pi}\sqrt{\frac{57.098}{231.41\times10^{-15}+50\times10^{-15}}}=0.022673\times10^8\,\unit{\Hz}=2267.3\,\unit{\kHz}$$
        \paragraph{}
        And with the simulation in COMSOL, the eigenfrequency($f_2$) changes to $2297.2\,\unit{\kHz}$. The simulation figure is shown below as Figure \ref{Fig.f2_increase_add_mass}.
        \begin{figure}[H]
            \centering
            \includegraphics[width=0.7\textwidth]{figures/f2_increase_add_mass}
            \caption{$f_2$ when increasing the added mass}
            \label{Fig.f2_increase_add_mass}
        \end{figure}
    \paragraph{}
    With the simulation in COMSOL, the tends of decreasing and increasing added mass is same with the calculation.
    By plugging in additional mass and frequency values, we can calculate the values of $m^*$ and $k_s$ in the COMSOL.\\
    Using $f=\frac{1}{2\pi}\sqrt{\frac{k_s}{m^*+m_b^*}}$ with the values below:
    \begin{equation}
        \begin{cases}
            m_b^*=12.5\,\unit{\pg}, f_2=2433.7\,\unit{\kHz}\\
            m_b^*=50\,\unit{\pg}, f_2=2297.2\,\unit{\kHz}
        \end{cases}
        \notag
    \end{equation}
    Thus, the values of $m^*$ and $k_s$ in the COMSOL are:
    \begin{equation}
        \begin{cases}
            m^*=293.9449461311249\times10^{-15}\,\unit{\kg}\\
            k_s=71.6549619115236\,\unit{\N/\m}
        \end{cases}
        \notag
    \end{equation}
    Also, using $m^*=\frac{2}{3} \rho Lwt$ and $k_s=\frac{2Ewt^3}{3L^3}$, we can calculate the Young's modulus and the density of silicon in COMSOL.
    % \begin{equation}
    %     \begin{cases}
    %         E=200.3428072923038GPa\\
    %         \rho =2920.81853339291kg/m^3
    %     \end{cases}
    %     \notag
    % \end{equation}
    The figure of comparing eigenfrequency in different added mass is plotted by Python below, 
    the figure in the left is the comparison between calculation and simulation with three groups of values, the figure in the right adds the simulation figure after recalculating:
    % \begin{figure}[H]
    %     \centering
    %     \includegraphics[width=0.7\textwidth]{figures/Comparing_eigenfrequency_between_calculation_and_simulation_in_different_added_mass.png}
    %     \caption{Comparing eigenfrequency in different added mass}
    %     \label{Fig.Comparing_eigenfrequency_in_different_added_mass}
    % \end{figure}
    % \begin{figure}[H]
    %     \centering
    %     \includegraphics[width=0.7\textwidth]{figures/Comparing_eigenfrequency_after_cal.png}
    %     \caption{Comparing eigenfrequency in different added mass after re-calculate}
    %     \label{Fig.Comparing_eigenfrequency_in_different_added_mass_after_re-calculate}
    % \end{figure}

    \begin{figure}[H]
        \centering
        \vspace{-0.35cm}
        \subfigtopskip=2pt
        \subfigbottomskip=2pt
        \subfigcapskip=-5pt
        \subfigure[Comparing eigenfrequency in different added mass]{
            \label{Fig.Comparing_eigenfrequency_in_different_added_mass}
            \includegraphics[width=0.32\linewidth]{figures/Comparing_eigenfrequency_between_calculation_and_simulation_in_different_added_mass.png}}
        \quad
        \subfigure[Comparing eigenfrequency after recalculating]{
            \label{Fig.Comparing_eigenfrequency_in_different_added_mass_after_recalculate}
            \includegraphics[width=0.32\linewidth]{figures/Comparing_eigenfrequency_after_calculate.png}}
        \label{Eigenfrequency1}
    \end{figure}
    The simulated values are close to the calculated values and have the same trend. 
    It can be considered that the design of the cantilever is reasonable and can be 
    used to measure the added mass based on the change of the resonance frequency.
    \subsubsection{Suggestions for improvements}
        \paragraph{}
        The microcantilever has been tested by increasing and decreasing the added mass, 
        which shows that this microcantilever has reliable performance. However, there are 
        still some ways to improve the performance of this microcantilever. Firstly, 
        the increasing and decreasing added mass values are not equal, increasing added 
        mass to $37.5\,\unit{\pg}$ could be better comparing with $50\,\unit{\pg}$. Secondly, it is useful to 
        improve sensitivity by the reduce the mass of microcantilever so that relative 
        change in mass is greater.

% --------------------------------------------------------
% Section 3: Conclusions
% --------------------------------------------------------
\section{Conclusions}
    This paper designs a microcantilever to measure added mass based on resonant 
    frequency and simulates by COMSOL. Comparing the results of calculation and 
    simulation, it is inferred that the Young's modulus and silicon density assumed 
    during calculation and simulation are different. As a result, the mass of the microcantilever $m^*$ and the parameters $k_s$ are simulated differently from the original calculation.
    And this paper tests the microcantilever by increasing and decreasing the added mass, which shows 
    that this microcantilever has reliable performance. The eigenfrequency of 
    calculation and simulation are close and have the same trend. Meanwhile, 
    the calculation and simulation results show that the resonant frequency 
    tends to decrease when the added mass increases, which is also consistent 
    with the calculated formula $f=\frac{1}{2\pi}\sqrt{\frac{k_s}{m^*+m_b^*}}$. 
    At the end, this paper puts forward some suggestions on how to improve this design and simulation.
    The microcantilever designed in this paper has the characteristics of micro size and weight and 
    high sensitivity, which is suitable for various fields requiring MEMS sensor devices.

\end{document}